\textbf{General rules}\\
\clause{The scope of adjustment and the object of application}

- This Regulation stipulates university and college education and training according to the credit system, including: training organization; Assess learning outcomes; graduation review and recognition.

- This regulation applies to all lecturers and students of training courses at university and college level at Ho Chi Minh City University of Technology - Vietnam National University.

\clause{Education program}
The educational program (EP) at the university and college level includes knowledge of two blocks:

- General education knowledge block (including subjects in the fields of social sciences - humanities, natural sciences and mathematics) to equip learners with: broad academic background; scientific worldview and correct human life view; understanding of nature, society and people; master scientific thinking methods; appreciate the cultural heritage of the nation and humanity; have ethics, awareness of civic responsibility; have the capacity to build and defend the country.

- The block of professional education knowledge is expressed in 2 groups of knowledge: the group of basic knowledge (basic knowledge of the industry or interdisciplinary) and the group of specialized knowledge - through which learners are equipped with the necessary knowledge. and vocational skills needed to enter the labor market.

The curriculum is a complete set of documents that include all the necessary information to organize training for a degree according to a specific program, industry/speciality. The most basic information in each EP:

- A list of all subjects, modules, certificates to be completed and, respectively, the total number of required credits accumulated per educational program, the standard average time and the maximum time to complete the content. content of this program

- The training program is a plan to arrange the subjects and the constituent modules of the educational program according to the sequence of semesters with the most essential constraints. The planned training program is developed by the functional unit of the school suitable for each form of training organization, and announced separately for each training course-industry/speciality, this is also the learning registration sequence that the school recommends. Reports should be followed by students in order to create the most favorable conditions for enrollment and knowledge acquisition.

Students must complete the subjects, modules and requirements as prescribed by the curriculum of the course-industry/major that the student is studying in order to be considered for graduation.

\clause{Subjects, courses and credits}
\subsection{Subjects}
A subject is a collection of knowledge about a particular area of expertise and is a complete unit that is taught and assessed within a semester. Learning activities are taught in a course include one or a combination of some of the following:

- Teaching theory: organized into subject classes;

- Teaching, practical guidance, exercises, discussions: organized in general classes or divided into separate groups;

- Teaching experiments, practicing in laboratories, workshops;

- Guide internships, graduate internships at external institutions;

- Guide projects, essay reports, large assignments;

- Guide and evaluate graduation theses, graduation essays; subject study and graduation exam.

Each subject has a subject number with a specified number of credits. Some subjects have prerequisite courses, pre-courses or parallel subjects (see Clause 6).

Each subject has a detailed outline showing the following basic contents: a summary of the course; prerequisite subjects, previous subjects, parallel subjects; main contents of chapters; textbooks, reference materials; course assessment (component scores and assessment rates); output standard; ... The course outline is approved and published together with the educational program and the planned training program. In case of adjustment, the course outline must be issued at least 2 weeks before the applicable semester.

Teachers must inform students about the regulations of the subject in the first lessons of the subject and publish it on the BKeL teaching support system along with the subject outline.

\subsection{Course}
A course is a combination of several subjects that are linked together to form a common knowledge block, and a certificate can be awarded upon completion of the course (such as a certificate of National Defense Education, Physical Education and Training,…).

Graduation modules are organized in one of the following formats: Graduation Thesis (LVTN) at the undergraduate level, Graduation Thesis (TLTN) at the college level, a combination of the Graduate Internship (TTTN) subject. with LVTN/TLTN, a combination of internship, a group of thematic subjects and a graduation exam.

\subsection{Credits}
Credit is a standard unit used to quantify the learning volume of students. One credit is equal to 15 theory lessons (equivalent to one class period/week in the main semester of 15 weeks); 30 periods of practice and experiment; 45-90 hours to visit the internship at the facility; 45-60 hours of writing essays, major assignments, projects, graduation theses, graduation theses. One lesson is equal to 50 minutes.

For theoretical or practical courses, experiments, to acquire one credit, students must spend an average of 15-30 hours in preparation and self-study (equivalent to 1-2 hours/week in the main semester 15 week). To prepare for the test, students need to spend at least 2/3 days for one credit.