\chapter{Training organizations}
\section{Students}
\subsection{Types of students}
Official students of HCMUT are students who have passed the corresponding entrance exam according to the enrollment regulations, have completed the admission procedures and have a decision to recognize students according to the course-industry/major, including:

-Regular university students (first degree): including mass programs and special programs (Talent Engineer Program, High Quality Engineer training program in Vietnam PFIEV (Vietnamese France), Advanced program, English program, ...)

- Regular college students

- Regular university students with a second degree

- University students work while studying (VLVH)

- University students in the form of Distance Learning (DTTXa)

- International affiliated students (study phase 1 at HCMUT, phase 2 at affiliated schools abroad and awarded degrees by affiliated schools)

In addition to official students, HCMUT also accepts exchange students for the purpose of supplementing short-term knowledge and does not accept university degrees. Exchange students include students studying at another university (domestic and foreign) and officials working in agencies, enterprises, and research institutes who wish to enroll in a limited number of universities. discipline of subjects.

To be admitted as an exchange student, a minimum qualification is required to ensure effective study. In some cases, students must pass a level test before they can study and be accepted by the Principal for a time limit for each specific case. Exchange students must fully comply with the regulations on registration of subjects like all other students, but they are not allowed to register for graduate courses. Tuition fees for exchange students are regulated at a separate rate similar to the level applied to the training program outside the obligation plan.

Exchange students are only granted certificates of learning results for the subjects studied, are not recognized as official students of the school and therefore are not entitled to the same social benefits and regimes as official students. Exchange students are not allowed to switch to the official system, and are not considered for graduation diplomas.

\subsection{Training year classification}
After each semester, students are classified the training year according to the number of accumulated credits (TCTL) as follows:

- First year students: less than 28 credits

- Second year students: from 28 to less than 56 credits

- Third year students: from 56 to less than 80 credits

- Fourth year students: from 80 to less than 128 credits

- Fifth year students (for training programs over four years): from 108 credits

\subsection{Change the form of training}
Full-time university students may be converted to part-time or remote training, even if the training period has expired but it has not been more than 10 years since the year they entered a regular program. Part-time student can be considered to switch to remote training.

\section{Training time and plan}
\subsection{Semesters}
Official semesters:

- Semester 1: includes 15 weeks of study and 2-3 weeks of course assessment

- Semester 2: includes 15 weeks of study and 2-3 weeks of course assessment

- Summer Semester: 8 weeks for internship subjects (TTTN/TT) outside the campus

In addition, extra semesters (optional) include 5-10 weeks of study and 1-3 weeks of subject assessment. In the extra semesters, project, internship (TTTN/TT), and graduation thesis (LVTN) subjects are not available.

Extra semesters:

- Overtime Semester: organize after-school subjects in parallel with semester 1 and semester 2

- Summer Semester: organize courses during the summer time

- Additional Semester which is held in the period between 2 main semesters

Exchange, transfer students (master's students), transfer students, ... are allowed to attend together (optional) with the main semester class and are considered to be in attendance.

The annual plan for organizing training activities is specified in the School Year Chart, which is organized by the Academic Affairs Office together with other units. The Principal of university will promulgate this plan for all levels and types of training in the university.

\subsection{Standard training time}
The standard training time is the number of semesters (Nhkc) designed so that an average student can complete the educational program of a discipline being held at the school, according to a training form and a specific type of degree (see Table~\ref{tab:training_time}).

Programs are standardized with 16 credits per semester. Affiliate and cooperation programs are converted according to the characteristics and origin of each program.

\subsection{Planned training time}
Planned training time (Nkh) is the total number of semesters designed according to the instructional plan for students enrolled in a training course in a particular form of training (see Table~\ref{tab:training_time}).

\subsection{Maximum training time}
- Maximum training time Nmax is the maximum number of main semesters for a student to attend HCMUT to complete the educational program (see Table 1). The starting time is counted from the time the student enrolls, especially for programs taught in English, it is counted from the time when the student meets the English standard for entry (but not more than one year from the time of admission).

- The semesters that are allowed to be suspended, and the semesters studied at other schools before transferring to HCMUT are included in the total training time.

\begin{table}[h]
	\caption{Training time}
	\label{tab:training_time}
	\begin{tabular}{|p{0.35\linewidth}|c|c|c|p{0.35\linewidth}|}
		\hline
		Types of Education & Nkh & Nhck & Nmax & Note \\ \hline
		\multirow{3}{6cm}{
Full-time university (1st degree)\\(From intake 2019 onwards)}  & 8    &  8    & 12     & Bachelor program     \\ \cline{2-5} 
		& 9-10    & 10     & 14     & Engineer program     \\ \cline{2-5} 
		&  10   &  12    &  16    & PFIEV program     \\ \hline
		\multirow{5}{6cm}{
Full-time university (1st degree)\\(From intake 2018 and earlier)}  &  8   &    9  &   13   &  
From intake 2014 (Only Advanced Program from intake 2009)    \\ \cline{2-5} 
		&  9   &   10   &  14    &   From intake 2009 to intake 2013   \\ \cline{2-5} 
		&  9   &  11    & 15     &  
		Particularly for Architecture major starts from intake 2014    \\ \cline{2-5} 
		& 10    & 12     & 16     &  Particularly for Architecture major from intake 2010 to intake z2013    \\ \cline{2-5} 
		&  10   &  12    & 16     &  PFIEV program    \\ \hline
	Exchange program	&   4  &  5    &  7    &      \\ \hline
		\multirow{2}{6cm}{Full-time university (2nd degree)}  &     &   6   &   10   &      \\ \cline{2-5} 
		&     &  7    &    11  &  Particularly for Civil engineering major    \\ \hline
Full-time university (college transfer to university)		&   4  & 5     &   8   &      \\ \hline
	Part-time university	&  9   & 10     & 20     &      \\ \hline
Remote university		&   9  &  10    & 20     &      \\ \hline
College		&    6 &   7   &  10    &      \\ \hline
	\end{tabular}
\end{table}

\subsection{Cases which are extended the maximum training time}
- Students who are suspended due to military service: the maximum study time is added to the number of semesters (up to 6 semesters) suspended due to military service (see Article 25).

- The following cases may extend the training period up to one semester:

+ Student is a foreigner;

+ Students are entitled to priority according to subjects (priority groups 1 and 2) or priority according to region 01 specified in the current regulations on enrollment of regular universities and colleges;

+ Internationally affiliated students already have a Letter of Acceptance from the affiliated school

+ Student is a person with a disability

+ Students who meet the graduation requirements on completing the educational program (number of TCTLN and score of TBTLN) but do not yet meet other conditions for consideration of graduation (standard English, social work) may extend the maximum training time by one semester for students to supplement the conditions to be considered for graduation (not allowed to register for subjects during this extended period).

+ Other special cases decided by the Principal.

When the maximum training time has expired, including the extended time, if the student is still not qualified to graduate, his/her name will be deleted because of the end of the study period. Regular students who have completed the maximum training period are allowed to drop out of school and switch to part-time training.

\subsection{Retest}
Students from intake 2013 and earlier (not applicable for intake 2014 and later) are extended one semester to participate in the final exam when all of the following conditions are satisfied simultaneously:

- Students who have passed the graduation course (LVTN/TLTN) and are in the courses that do not have a second exam (see Article 12).

- The subjects registered for the final exam must be the subjects that have been studied and have non-zero assessment results (absence or prohibition will be counted as zero points).

- Satisfy all other requirements set forth by the academic board and published in the final exam announcement each semester.

Students must submit an application for the final exam in accordance with regulations and can only register for the final exam once in the Nmax +1 semester for all outstanding subjects in the curriculum of the course-industry/specialization. Cases that do not register for the exam or register incorrectly will not be considered for any reason.

\section{Types of subjects and courses}
\subsection{Compulsory subjects, compulsory courses}
These are the subjects and courses containing the main knowledge content of the curriculum. Students must accumulate required subjects and modules to be considered for graduation or transfer.

\subsection{Subjects and groups of electives}
These are the subjects in the curriculum that contain the necessary content for one of the many career orientations (industry / major / professional direction) that students can freely choose to enroll in.

In the curriculum, elective subjects are grouped with specific regulations on the minimum number of credits that must be accumulated for each group. To be eligible for graduation, students must earn the minimum number of credits specified for the elective subject group. See also the rules on how to calculate the GPA and the number of accumulated credits (Article 19).

\subsection{Prerequisite subjects, pre-subjects, parallel subjects}
Subject A is a prerequisite of subject B: a mandatory condition to register for subject B is to have studied and passed subject A.

- Subject A is a previous subject of subject B: a mandatory condition to register for subject B is that you have studied and have a different final score of subject A (prohibition on CT exam, absence of VT exam and failure to pass KD) is counted as 0.

- Subject A is a parallel subject of a subject B: a mandatory condition to register for subject B is to have registered for subject A. Students are allowed to register for subject B in the same semester as they have enrolled. subject A or in the following semesters.

\subsection{Equivalent subjects, alternative subjects}
Equivalence subject is a subject (or group of subjects) belonging to the educational curriculum of a course-industry/major whose content is equivalent (or covers) with the subject (or group of subjects) in the educational curriculum of the course-industry/major. is different. Substitute subject is a subject (or group of subjects) belonging to the educational curriculum of a course-industry/major with similar content and can be substituted for a subject (or group of subjects) belonging to the educational curriculum of the course-industry/specialty. other branches that no longer organize teaching. The list of equivalent and replacement subjects is a supplement to the curriculum of the curriculum during operation and is notified and updated every semester.

Students need to register for courses according to the subject code of the curriculum of the course-industry/speciality they are studying (original curriculum). In case the student passes the equivalent or substituted subject, the subject of the original education program will be scored for exemption (MT) and counted in the number of accumulated industry credits but not included in the cumulative average of the industry.

See also the regulations on transcripts, academic records (15.4) exempted points, number of credits considered for exemption (17.4), how to calculate the average score and the number of accumulated credits (Article 18, Article 19).

\noindent\textbf{Instructions for equivalent or alternative subjects:}

- From intake 2014, a new training program is built. The list of equivalent subjects between the old training curriculum (from intake 2008 to intake 2013, especially the part-time system from intake 2009 to intake 2014) and the new training curriculum (applicable from intake 2014, exclusively from intake 2015) provided by Faculty regulations are as follows (continued to be updated).

- Students should register for courses according to the subject code of the training program of the course they are studying. The school will continue to open the subjects of the old curriculum when students have a need to register and open classes.

- In case students study equivalent subjects instead of studying the correct subjects according to the original curriculum, they will be handled as follows:

+ The equivalent subject studied is recorded on the transcript, calculated into the semester average and calculated into the cumulative GPA, not the cumulative GPA of the major.

+ If the student passes the equivalent subject, the subject of the original training program will automatically record the exempt score (MT) in the transfer/reservation point, which will be counted in the accumulated credits but not in the semester average, the average score cumulative average of the major.

For example, subject A in the old curriculum is equivalent to subject B in the new curriculum. When students of the old training program studying subject B get 8 points in semester 2/15-16, the transcript will be:

\noindent\textbf{Semester 2/15-16}\\
Subject B 			~~~8\\
\textbf{Transfer/reservation points}\\
Subject A			~~~MT\\

Subject B is included in the semester grade point average, in the cumulative GPA but not in the industry cumulative GPA.

Subject A counts toward cumulative credits but not semester GPA, cumulative GPA, and cumulative major GPA.

Similarly for students of new training program who have registered for subject A.

If 2 subjects in the old curriculum (including 1 theory and 1 experiment) are equivalent to 1 subject in the new curriculum and the student fails to pass 1 of the 2 old subjects, they must retake that subject.

\subsection{Equivalence between training forms}
Regular students can only study and accumulate open subjects for the regular system (based on subject groups). Subjects belonging to special programs (High Quality, Advanced, Exchange, ...) are considered regular.

- Students of the full-time program can study and accumulate subjects in the subject groups of the High Quality, Advanced, Exchange programs provided they have an English level of IELTS 6.0 or equivalent.

- Students of High-quality, Advanced, Exchange programs can only register for subjects in the right group of high-quality, advanced, international-linked programs. The form of registration for transfer is not applied to study with the mass mainstream, including subjects taught in Vietnamese.

\noindent\textbf{Time of application}

\begin{itemize}
	\item Intake 2019 and courses with intake 2019 onwards: from Semester 192
	
	\item Intake 2018 and earlier: from Semester 202
\end{itemize}

\noindent\textbf{Special cases}

\begin{itemize}
	\item In some special cases, students of High Quality, Advanced, Exchange programs are approved, allowing them to register for courses in subject groups of the regular program when students need to repay their loans. urgent subject to transfer or graduate
	
	\item Students study in the same regular class but still pay tuition fees according to the regulations of High Quality, Advanced, Exchange programs.
\end{itemize}

\noindent\textbf{For the cases considered to study with the regular system}
\begin{itemize}
	\item If in the original curriculum, these subjects are taught in Vietnamese: they will be included in the number of TCTLN and TDBTLN.
	
	\item If in the original curriculum, these subjects are taught in English: they will be counted in the number of NCDs and a maximum of 2 subjects in the case of re-learning/improvement.
	
	\item Other cases: not included in the number of TCTLN and TDBTLN.
	
\end{itemize}

- Part-time students can study and accumulate open subjects for part-time and full-time at the same level of study organized at the school and at affiliated institutions.

- Remote students can learn and accumulate open subjects for the form of training in part-time, remote and full-time at the same level of study, organized at the school and at affiliated institutions.

\section{Tuition}
\subsection{How to calculate tuition fees}
Tuition is calculated based on the number of credits of the course, the number of course periods, except for some special cases, or calculated by semester/school year.

From intake 2018 onwards, the tuition fee of each semester is calculated according to the number of credits. Tuition is calculated separately for the main semester, summer semester, sub-semester, audition and is calculated as the total number of student credits registered in that semester multiplied by the fee for one credit and plus the tuition fee for the student. subjects have their own rules.

From the academic year 2019, the tuition fee of the main semesters is calculated at 50\% of the tuition fee per academic year. If the student exceeds the maximum number of design credits for each (main) semester (see 9.2), the excess credits will be calculated according to the number of credits. In special cases, when students are allowed to study less than the number of design credits, students can be deducted tuition fees. Tuition fees for the semesters are calculated based on the number of credits.

Tuition fees for a credit, regulations on tuition fees for projects, internships, dissertations, ... and subjects with separate tuition fees are considered and regulated by the school's principal. education level, training system, training form for each semester. Tuition fees for special programs are regulated separately.”

\subsection{Tuition payment}
Students must pay tuition fees on time for registered courses and/or have official TKB. Students check and pay tuition fees through the school's online payment gateway. Unpaid tuition fees will be debited. For the main semester tuition, students must pay the tuition fee at least 50\% before the middle of the semester and pay the rest before the last week of the semester (specific deadline is announced each semester). Students who pay late or do not pay will be deducted training points (minus 5 points / time of violation). Students who still do not complete the tuition fee after the deadline in the semester will be suspended from studying, canceled the official course registration results / TKB and academic results (if any) of the semester.

For the tuition fee for the semester and the audition, students must pay the full tuition fee before the 3rd week of the semester (specific deadline is announced each semester). Students who pay late or fail to pay will have their timetable deleted, their name will not be able to attend the test/exam, and they will not be able to register for the next semester. In this case, the student must pay the tuition fee (even though it is late) and submit an application stating the reason for the delay in order to be considered for the exam and to be registered for the next semester. Repeated violations will not be considered.

The students who are considered for exemption or reduction of tuition fees according to the policy should contact Student Affairs Office to make a dossier in accordance with regulations. In case of unexpected difficulties, students need to contact Student Affairs Office to apply for a postponement of tuition payment. If considered deferred payment of tuition fees, students must pay in the next semester this tuition fee together with the tuition fee of the next semester. No postponement for 2 consecutive semesters is considered. Only consider postponement, exemption or reduction of tuition fees for regular students by 1 in the main semesters, not for extra semesters and auditions.

Particularly for Advanced, High quality, and Exchange programs:

- Students must pay the full tuition fee according to the specific deadline announced in each semester.

- Students who have not completed the tuition fee after the payment deadline:
\begin{itemize}
	\item For the main semester or summer auditions: Suspended from school; canceled all results of subject registration (including the same subjects of the mass regular); scholarships are not considered; not be granted a certificate of student; 10 training points deducted.
	
	\item For the extra semester in the main semester: The schedule of the registered subjects will be deleted; remove the name from the checklist/exam and not allow the test for the next extra semester; tuition fees debited; scholarships are not considered; not be granted a certificate of student; 10 training points deducted.

\end{itemize}

\section{Student class and subject class}
\subsection{Student class}
Organized by faculty, training course and by a teacher in charge. Teachers also take on the role of advising students in the class on academic issues, helping students to plan their study activities in each semester and prepare a plan for the entire training course. The organization of activities of the student class, the roles and responsibilities of the homeroom teacher are specified in the Student Work Regulations of HCMUT.

\subsection{Subject class}
A class of students who register for the same subject, have the same timetable in a semester, and have the subject class name. Subjects are opened according to the curriculum of the curriculum and according to the needs of each semester. Project subjects, graduation thesis (LVTN, TLTN) are opened in each main semester.

The number of students in a subject class is limited by the capacity of the classroom, laboratory or arranged according to the specific requirements of the subject. The average number of students in a subject class for all types of subjects is as follows, except for special cases specified separately:

- Political and general subjects: 140 students. Particularly, the subjects in the lecture hall are calculated according to the capacity of the lecture hall.

- Basic subjects of the whole faculty: 100 - 120 students

- Basic subjects: 60 - 80 students

- Specialized subjects, foreign languages, Physical education, Introductory technical subjects: 40 – 60 students

- Subjects under the Advanced, High-Quality, Exchange programs: 30-45 students (for Political subjects: 60-80 students)

The minimum number of students to organize a class is 60 students with general and general subjects; 15 students with narrow specialized subjects; 30 students with other subjects. The Academic Affairs Office and the Office for International Study Programs will consider opening classes in other special cases. Allow Faculties to request the maximum number of students for a group of subjects that can exceed the above regulations.

\section{Subject registration}
\subsection{Subject registration}
Students make course registration according to the process and duration in the Course Registration Notice of each semester published on the Training Department website.

In each semester of study at the school, students must register for the subject and have an official schedule. Students who do not have a timetable for the main semesters will be forced to suspend their studies (or students apply for a temporary suspension) and will not be issued a certificate of studenthood (see 26.2).

During the summer period, it is possible to organize internship subjects in accordance with the official teaching plan and at the same time organize the summer semester (not required) for students to register for subjects as needed.

Students must register their own subjects at each end of the semester to have an official timetable for the next semester, especially students in year 1 are set a fixed timetable for the first semester. Students need to carefully monitor the course registration schedule of the semesters and follow the correct process, on time.

- For the main semester: usually the registration of the subject is made in May for semester 1, in November for semester 2.

- For sub-semesters: usually the registration of the subject is made in September for the 1th semester of the auditorium, in January for the 2th semester and in May for the summer semester. Students are not allowed to register for the semester without the main semester schedule, special cases are considered separately.

In order to register for a subject, students must meet the prerequisites, pre-course subjects, parallel subjects and other binding conditions.
In one semester, including the main semester and the accompanying sub-semesters, only one subject is allowed in a single session.  Two grade summaries of the same subject code cannot be allowed in the same semester transcript.

\subsection{Design credits for one main semester}
The number of credits designed for a major semester is designed in educational program in the direction of evenly distributing subjects throughout the training period and ensuring the study and working time of students in accordance with the working time as prescribed by law. From intake 2019, the maximum number of design credits for a primary semester is 17 credits.

\subsection{Maximum credits for one main semester}
The maximum number of credits allowed to register in a primary semester (TCmax) is 21 credits. Separate cases:

- Students are quite good ($ GPA \ge 7.5 $ or talent-engineering program students): TCmax = 25 credits (Not applicable to semester with graduation internship and graduation thesis) and must be approved by the Faculty.

- Students of Vietnam-France High Quality Engineer (PFIEV) program from intake 2018 to the past: ): TCmax = 35 credits.

- Classes with an out-of-hours main semester are not limited to the Tcmax but are limited by the time fund that can be scheduled.

\subsection{Minimum number of credits/subjects in a major semester}
