\chapter{Training organizations}
\section{Students}
\subsection{Types of students}
Official students of HCMUT are students who have passed the corresponding entrance exam according to the enrollment regulations, have completed the admission procedures and have a decision to recognize students according to the course-industry/major, including:

-Regular university students (first degree): including mass programs and special programs (Talent Engineer Program, High Quality Engineer training program in Vietnam PFIEV (Vietnamese France), Advanced program, English program, ...)

- Regular college students

- Regular university students with a second degree

- University students work while studying (VLVH)

- University students in the form of Distance Learning (DTTXa)

- International affiliated students (study phase 1 at HCMUT, phase 2 at affiliated schools abroad and awarded degrees by affiliated schools)

In addition to official students, HCMUT also accepts exchange students for the purpose of supplementing short-term knowledge and does not accept university degrees. Exchange students include students studying at another university (domestic and foreign) and officials working in agencies, enterprises, and research institutes who wish to enroll in a limited number of universities. discipline of subjects.

To be admitted as an exchange student, a minimum qualification is required to ensure effective study. In some cases, students must pass a level test before they can study and be accepted by the Principal for a time limit for each specific case. Exchange students must fully comply with the regulations on registration of subjects like all other students, but they are not allowed to register for graduate courses. Tuition fees for exchange students are regulated at a separate rate similar to the level applied to the training program outside the obligation plan.

Exchange students are only granted certificates of learning results for the subjects studied, are not recognized as official students of the school and therefore are not entitled to the same social benefits and regimes as official students. Exchange students are not allowed to switch to the official system, and are not considered for graduation diplomas.

\subsection{Training year classification}
After each semester, students are classified the training year according to the number of accumulated credits (TCTL) as follows:

- First year students: less than 28 credits

- Second year students: from 28 to less than 56 credits

- Third year students: from 56 to less than 80 credits

- Fourth year students: from 80 to less than 128 credits

- Fifth year students (for training programs over four years): from 108 credits

\subsection{Change the form of training}
Full-time university students may be converted to part-time or remote training, even if the training period has expired but it has not been more than 10 years since the year they entered a regular program. Part-time student can be considered to switch to remote training.

\section{Training time and plan}
\subsection{Semesters}
Official semesters:

- Semester 1: includes 15 weeks of study and 2-3 weeks of course assessment

- Semester 2: includes 15 weeks of study and 2-3 weeks of course assessment

- Summer Semester: 8 weeks for internship subjects (TTTN/TT) outside the campus

In addition, extra semesters (optional) include 5-10 weeks of study and 1-3 weeks of subject assessment. In the extra semesters, project, internship (TTTN/TT), and graduation thesis (LVTN) subjects are not available.

Extra semesters:

- Overtime Semester: organize after-school subjects in parallel with semester 1 and semester 2

- Summer Semester: organize courses during the summer time

- Additional Semester which is held in the period between 2 main semesters

Exchange, transfer students (master's students), transfer students, ... are allowed to attend together (optional) with the main semester class and are considered to be in attendance.

The annual plan for organizing training activities is specified in the School Year Chart, which is organized by the Academic Affairs Office together with other units. The Principal of university will promulgate this plan for all levels and types of training in the university.

\subsection{Standard training time}
The standard training time is the number of semesters (Nhkc) designed so that an average student can complete the educational program of a discipline being held at the school, according to a training form and a specific type of degree (see Table 1).

Programs are standardized with 16 credits per semester. Affiliate and cooperation programs are converted according to the characteristics and origin of each program.

\subsection{Planned training time}
Planned training time (Nkh) is the total number of semesters designed according to the instructional plan for students enrolled in a training course in a particular form of training (see Table 1).

\subsection{Maximum training time}
- Maximum training time $ \textbf{N}_{max} $ is the maximum number of main semesters for a student to attend HCMUT to complete the educational program (see Table 1). The starting time is counted from the time the student enrolls, especially for programs taught in English, it is counted from the time when the student meets the English standard for entry (but not more than one year from the time of admission).