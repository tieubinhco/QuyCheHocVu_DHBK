\chapter{Training organizations}
\section{Students}
\subsection{Types of students}
Official students of HCMUT are students who have passed the corresponding entrance exam according to the enrollment regulations, have completed the admission procedures and have a decision to recognize students according to the course-industry/major, including:

-Regular university students (first degree): including mass programs and special programs (Talent Engineer Program, High Quality Engineer training program in Vietnam PFIEV (Vietnamese France), Advanced program, English program, ...)

- Regular college students

- Regular university students with a second degree

- University students work while studying (VLVH)

- University students in the form of Distance Learning (DTTXa)

- International affiliated students (study phase 1 at HCMUT, phase 2 at affiliated schools abroad and awarded degrees by affiliated schools)

In addition to official students, HCMUT also accepts exchange students for the purpose of supplementing short-term knowledge and does not accept university degrees. Exchange students include students studying at another university (domestic and foreign) and officials working in agencies, enterprises, and research institutes who wish to enroll in a limited number of universities. discipline of subjects.

To be admitted as an exchange student, a minimum qualification is required to ensure effective study. In some cases, students must pass a level test before they can study and be accepted by the Principal for a time limit for each specific case. Exchange students must fully comply with the regulations on registration of subjects like all other students, but they are not allowed to register for graduate courses. Tuition fees for exchange students are regulated at a separate rate similar to the level applied to the training program outside the obligation plan.

Exchange students are only granted certificates of learning results for the subjects studied, are not recognized as official students of the school and therefore are not entitled to the same social benefits and regimes as official students. Exchange students are not allowed to switch to the official system, and are not considered for graduation diplomas.

\subsection{Training year classification}
After each semester, students are classified the training year according to the number of accumulated credits (TCTL) as follows:

- First year students: less than 28 credits

- Second year students: from 28 to less than 56 credits

- Third year students: from 56 to less than 80 credits

- Fourth year students: from 80 to less than 128 credits

- Fifth year students (for training programs over four years): from 108 credits

\subsection{Change the form of training}
Full-time university students may be converted to part-time or remote training, even if the training period has expired but it has not been more than 10 years since the year they entered a regular program. Part-time student can be considered to switch to remote training.

\section{Training time and plan}
\subsection{Semesters}
Official semesters:

- Semester 1: includes 15 weeks of study and 2-3 weeks of course assessment

- Semester 2: includes 15 weeks of study and 2-3 weeks of course assessment

- Summer Semester: 8 weeks for internship subjects (TTTN/TT) outside the campus

In addition, extra semesters (optional) include 5-10 weeks of study and 1-3 weeks of subject assessment. In the extra semesters, project, internship (TTTN/TT), and graduation thesis (LVTN) subjects are not available.

Extra semesters:

- Overtime Semester: organize after-school subjects in parallel with semester 1 and semester 2

- Summer Semester: organize courses during the summer time

- Additional Semester which is held in the period between 2 main semesters

Exchange, transfer students (master's students), transfer students, ... are allowed to attend together (optional) with the main semester class and are considered to be in attendance.

The annual plan for organizing training activities is specified in the School Year Chart, which is organized by the Academic Affairs Office together with other units. The Principal of university will promulgate this plan for all levels and types of training in the university.

\subsection{Standard training time}
The standard training time is the number of semesters (Nhkc) designed so that an average student can complete the educational program of a discipline being held at the school, according to a training form and a specific type of degree (see Table~\ref{tab:training_time}).

Programs are standardized with 16 credits per semester. Affiliate and cooperation programs are converted according to the characteristics and origin of each program.

\subsection{Planned training time}
Planned training time (Nkh) is the total number of semesters designed according to the instructional plan for students enrolled in a training course in a particular form of training (see Table~\ref{tab:training_time}).

\subsection{Maximum training time}
- Maximum training time Nmax is the maximum number of main semesters for a student to attend HCMUT to complete the educational program (see Table 1). The starting time is counted from the time the student enrolls, especially for programs taught in English, it is counted from the time when the student meets the English standard for entry (but not more than one year from the time of admission).

- The semesters that are allowed to be suspended, and the semesters studied at other schools before transferring to HCMUT are included in the total training time.

\begin{table}[h]
	\caption{Training time}
	\label{tab:training_time}
	\begin{tabular}{|p{0.35\linewidth}|c|c|c|p{0.35\linewidth}|}
		\hline
		Types of Education & Nkh & Nhck & Nmax & Note \\ \hline
		\multirow{3}{6cm}{
Full-time university (1st degree)\\(From intake 2019 onwards)}  & 8    &  8    & 12     & Bachelor program     \\ \cline{2-5} 
		& 9-10    & 10     & 14     & Engineer program     \\ \cline{2-5} 
		&  10   &  12    &  16    & PFIEV program     \\ \hline
		\multirow{5}{6cm}{
Full-time university (1st degree)\\(From intake 2018 and earlier)}  &  8   &    9  &   13   &  
From intake 2014 (Only Advanced Program from intake 2009)    \\ \cline{2-5} 
		&  9   &   10   &  14    &   From intake 2009 to intake 2013   \\ \cline{2-5} 
		&  9   &  11    & 15     &  
		Particularly for Architecture major starts from intake 2014    \\ \cline{2-5} 
		& 10    & 12     & 16     &  Particularly for Architecture major from intake 2010 to intake z2013    \\ \cline{2-5} 
		&  10   &  12    & 16     &  PFIEV program    \\ \hline
	Exchange program	&   4  &  5    &  7    &      \\ \hline
		\multirow{2}{6cm}{Full-time university (2nd degree)}  &     &   6   &   10   &      \\ \cline{2-5} 
		&     &  7    &    11  &  Particularly for Civil engineering major    \\ \hline
Full-time university (college transfer to university)		&   4  & 5     &   8   &      \\ \hline
	Part-time university	&  9   & 10     & 20     &      \\ \hline
Remote university		&   9  &  10    & 20     &      \\ \hline
College		&    6 &   7   &  10    &      \\ \hline
	\end{tabular}
\end{table}

\subsection{Cases which are extended the maximum training time}
- Students who are suspended due to military service: the maximum study time is added to the number of semesters (up to 6 semesters) suspended due to military service (see Article 25).

- The following cases may extend the training period up to one semester:

+ Student is a foreigner;

+ Students are entitled to priority according to subjects (priority groups 1 and 2) or priority according to region 01 specified in the current regulations on enrollment of regular universities and colleges;

+ Internationally affiliated students already have a Letter of Acceptance from the affiliated school

+ Student is a person with a disability

+ Students who meet the graduation requirements on completing the educational program (number of TCTLN and score of TBTLN) but do not yet meet other conditions for consideration of graduation (standard English, social work) may extend the maximum training time by one semester for students to supplement the conditions to be considered for graduation (not allowed to register for subjects during this extended period).

+ Other special cases decided by the Principal.

When the maximum training time has expired, including the extended time, if the student is still not qualified to graduate, his/her name will be deleted because of the end of the study period. Regular students who have completed the maximum training period are allowed to drop out of school and switch to part-time training.

\subsection{Retest}
Students from intake 2013 and earlier (not applicable for intake 2014 and later) are extended one semester to participate in the final exam when all of the following conditions are satisfied simultaneously:

- Students who have passed the graduation course (LVTN/TLTN) and are in the courses that do not have a second exam (see Article 12).

- The subjects registered for the final exam must be the subjects that have been studied and have non-zero assessment results (absence or prohibition will be counted as zero points).

- Satisfy all other requirements set forth by the academic board and published in the final exam announcement each semester.

Students must submit an application for the final exam in accordance with regulations and can only register for the final exam once in the Nmax +1 semester for all outstanding subjects in the curriculum of the course-industry/specialization. Cases that do not register for the exam or register incorrectly will not be considered for any reason.

\section{Types of subjects and courses}
\subsection{Compulsory subjects, compulsory courses}
These are the subjects and courses containing the main knowledge content of the curriculum. Students must accumulate required subjects and modules to be considered for graduation or transfer.

\subsection{Subjects and groups of electives}
These are the subjects in the curriculum that contain the necessary content for one of the many career orientations (industry / major / professional direction) that students can freely choose to enroll in.

In the curriculum, elective subjects are grouped with specific regulations on the minimum number of credits that must be accumulated for each group. To be eligible for graduation, students must earn the minimum number of credits specified for the elective subject group. See also the rules on how to calculate the GPA and the number of accumulated credits (Article 19).

\subsection{Prerequisite subjects, pre-subjects, parallel subjects}
Subject A is a prerequisite of subject B: a mandatory condition to register for subject B is to have studied and passed subject A.

- Subject A is a previous subject of subject B: a mandatory condition to register for subject B is that you have studied and have a different final score of subject A (prohibition on CT exam, absence of VT exam and failure to pass KD) is counted as 0.

- Subject A is a parallel subject of a subject B: a mandatory condition to register for subject B is to have registered for subject A. Students are allowed to register for subject B in the same semester as they have enrolled. subject A or in the following semesters.

\subsection{Equivalent subjects, alternative subjects}
Equivalence subject is a subject (or group of subjects) belonging to the educational curriculum of a course-industry/major whose content is equivalent (or covers) with the subject (or group of subjects) in the educational curriculum of the course-industry/major. is different. Substitute subject is a subject (or group of subjects) belonging to the educational curriculum of a course-industry/major with similar content and can be substituted for a subject (or group of subjects) belonging to the educational curriculum of the course-industry/specialty. other branches that no longer organize teaching. The list of equivalent and replacement subjects is a supplement to the curriculum of the curriculum during operation and is notified and updated every semester.

Students need to register for courses according to the subject code of the curriculum of the course-industry/speciality they are studying (original curriculum). In case the student passes the equivalent or substituted subject, the subject of the original education program will be scored for exemption (MT) and counted in the number of accumulated industry credits but not included in the cumulative average of the industry.

See also the regulations on transcripts, academic records (15.4) exempted points, number of credits considered for exemption (17.4), how to calculate the average score and the number of accumulated credits (Article 18, Article 19).